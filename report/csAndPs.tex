\section{Channels and Processes}
\label{sec:channels}

Process modelling languages are based upon the principles of processes communicating across channels. As such this part of our implementation is fairly key. 

\subsection{Channels}

As described in section \ref{subsec:types}, our data type representing a Channel is as follows:

\begin{minted}{hs}
data Channel = Channel {
               send         :: String -> IO ()
             , receive      :: IO String
             , extra        :: [String]
             }
\end{minted}

In other words, a $Channel$ is modelled as an object with a $send$ function, which takes a string and sends it somewhere, a $receive$ function, which returns a string in the IO monad, and some other data, $extra$, as a list of strings.

\subsubsection{Serialising Channels}
\label{sec:serialisingChannels}
\subsection{Processes}
\label{sec:processes}

