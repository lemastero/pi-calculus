\section{Plan}

Currently, we have completed most of our background research, and have started building a small web server in Haskell. The purpose of the latter is firstly to learn more about web frameworks in general, and secondly, once we have built a basic implementation, we can start to see how easy it will be to interact with conventional servers.

\subsection{Milestone Dates}

There are roughly 16 weeks between now and 17$^{th}$ June, the preliminary archive submission deadline. We therefore aim to split that time into eight periods of two weeks, and at the end of each of those periods we would like to achieve the following:
\subsubsection{7$^{th}$ March} 
By this stage we would ideally have a small framework set up against which we could start testing some of the basic features of the language. We will start to write up the implementation details for this framework.
Basic interaction with small servers i.e. \verb!in(a,message);out(a,message)!   
\subsubsection{21$^{st}$ March}
We would like to be able to compile a subset of the language, definitely \verb?in, out, !, |, if then else? and be able to test some basic interactions. If this is not possible at this stage, we may have to rethink the restrictions we have set for the language. Continue to write up implementation details for whatever we have achieved so far. 
\subsubsection{4$^{th}$ April}  
Ideally we should be able to model and execute a basic handshake protocol by this stage, this will have required the addition of \verb!new! and \verb!let! to our compile-able subset, and also a few functions. Continue to add any further implementation details.
\subsubsection{18$^{th}$ April}
If all goals met so far, begin trying to speed up compilation times and responsiveness of compiled programs. If not then work out what is causing difficulties, maybe rethink strategy/implementation. Write up anything relevant on either changing the implementation or increasing responsiveness.
\subsubsection{2$^{nd}$ May}    
Hopefully have an acceptably responsive (comparable to similar implementations of models in existing languages) model. Start polishing the compiler and working towards complete language compilation. Begin writing evaluation, by this stage how much is possible in the time left will be very clear.
\subsubsection{16$^{th}$ May} 
If the compiler is not yet complete continue working on it. If it is, then continue work on making the implementation more responsive. 
\subsubsection{19$^{th}$ - 23$^{rd}$ May : Health Check-Up}     
\subsubsection{30$^{th}$ May}     
At the very least 70\% of the report should be done, and any remaining parts should have a clear structure laid out. Continue tinkering if necessary, if not then keep writing the report.
\subsubsection{13$^{th}$ June}     
Address any final issues with the compiler. If in working order, attempt some complex models. Assess any major issues faced during the course of the project
Finish evaluation and conclusion
\subsubsection{17$^{th}$ June - Project submission deadline} 
\subsubsection{23$^{rd}$ June - Preliminary Archive Submission Deadline}     
\subsubsection{30$^{th}$ June - Final Project Archive Submission Deadline}     

