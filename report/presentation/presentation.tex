\documentclass[t]{beamer}
\usetheme{iclpt}
\usepackage{minted}
\usepackage{hyperref}

\author{Will de Renzy-Martin}
\title{phi$>$ The Applied $\pi$-calculus Interpreter}

\begin{document}
\frame{\maketitle}

\begin{frame}
\frametitle{Handshake Protocol in Plain English}
\begin{itemize}
    \item \href{http://tools.ietf.org/html/rfc5246\#section-7.4}{RFC Specification}
\end{itemize}
\end{frame}

\begin{frame}[fragile]
\frametitle{Handshake Protocol in phi}
\begin{minted}{ocaml}
let clientA(pkA,skA,pkB) =  
    (out(c,pkA); in(c,x); 
    let y = adec(skA,x) in
    let pair(pkD,j) = getmsg(y) in if pkD = pkB then 
                                        let k = j in
    out(c,senc(k,s)))
let serverB(pkB,skB) = 
    in(c',pkX);
    new k;
    out(c',aenc(pkX,sign(pkB,pair(pkB,k))));
    in(c',x);
    let z = sdec(k,x) in out(stdout,z)
\end{minted}

\end{frame}

\begin{frame}
\frametitle{Process Calculi}
\begin{itemize}
        \pause
    \item Communication
        \pause
    \item Sequential Composition
        \pause
    \item Parallel Composition
        \pause
    \item Reduction Semantics
        \pause
    \item Hiding 
        \pause
    \item Recursion and Replication
        \pause
    \item The Null Process
\end{itemize}
\end{frame}

\begin{frame}[fragile]
\frametitle{phi}
\begin{table}[hc!]
    \begin{tabular}{l l}
        \verb!0!                      &  Null Process\\
        \verb!in(u,M)!                &  Message in\\
        \verb!out(u,M)!               &  Message out\\
        \verb!P | Q!                  &  Parallel Composition\\
        \verb!P ; Q!                  &  Sequential Composition\\
        \verb?!P?                     &  Replication\\
        \verb!new x!                  &  Name reservation\\
        \verb!if p(M) then P else Q!  &  Conditionals\\
        \verb!let X = M in P!         &  Hiding / Pattern Matching \\
        \verb!&t!                     &  Function call\\
    \end{tabular}
\end{table} 
\end{frame}

\begin{frame}[fragile]
\frametitle{Processes}
\begin{minted}{hs}
eval env (Conc procs)  = do
                var <- liftIO newEmptyMVar 
                mapM_ (forkProcess var) procs
                res <- liftIO (takeMVar var)
                case res of
                    Left err -> throwE err
                    Right _  -> return ()
        where
            forkProcess var proc = liftIO $ forkIO $ do
                        res <- runExceptT (eval env proc)
                        _ <- tryPutMVar var res
                        return ()
\end{minted}
\end{frame}

\begin{frame}[fragile]
\frametitle{Channels}
\begin{minted}{hs}
data Channel = Channel {
               send         :: String -> IO ()
             , receive      :: IO String
             , extra        :: [String]
             }
\end{minted}

\end{frame}


\begin{frame}[fragile]
\frametitle{Primitives}
\begin{minted}{hs}
primitives :: [(Name          , TermFun)]
primitives = [ ("fst"         , first)
             , ("snd"         , secnd)
             , ("hash"        , hash)
             , ("getmsg"      , getmsg)
             , ("sdec"        , sdec)
             , ("senc"        , binaryId "senc")
             , ("adec"        , adec)
             , ("aenc"        , binaryId "aenc")
             , ("sign"        , binaryId "sign")
             , ("checksign"   , checksign)
             , ("mac"         , mac) ..]
\end{minted}
\end{frame}

\begin{frame}[fragile]
\frametitle{Pattern Matching}
\begin{minted}{ocaml}
let ls = list(1,pair(2,list(3,4,5)),6) in 
    let list(_1,pair(_2,list(_3,x,_5)),_6) = ls in 
        out(stdout,x)
\end{minted}
\end{frame}

\begin{frame}[fragile]
\frametitle{Pattern Matching}
\begin{minted}{ocaml}
let follow(ch,r) = (out(ch,r);in(ch,resp:HttpResponse);
  let list(c,_,_,b) = resp in
    if c = 302 
      then let req = httpReq(getHeader("location",resp),
                                       headers(),
                                       httpGet()) in 
          &follow(ch,req) 
      else out(stdout,b)) 
\end{minted}
\end{frame}

\begin{frame}
\frametitle{Limitations}
\begin{itemize}
        \pause
    \item "Single Shot" Socket Channels
        \pause
    \item Cryptographically insecure 
        \pause
    \item Hacks
\end{itemize}

\end{frame}

\begin{frame}[fragile]
\frametitle{Installing}
    The source is available on Hackage, and can be installed using cabal:
    \begin{minted}{shell-session}
        cabal update
        cabal install pi-calculus
    \end{minted}

    Alternatively you can clone the source and build using cabal:

    \begin{minted}{shell-session}
        git clone git@github:renzyq19/pi-calculus
        cd pi-calculus/pi
        cabal install
    \end{minted}
\end{frame}


\begin{frame}
\frametitle{Demonstrations}
\begin{itemize}
\pause
    \item Simple Client - Server
\pause
    \item Secret Channel Client - Server
\pause
    \item Follow Redirects
\end{itemize}
\end{frame}


\begin{frame}
\frametitle{Wrap Up}
\begin{itemize}
\pause
    \item Questions?
\end{itemize}
\end{frame}


\end{document}
